\documentclass{beamer}


\usepackage{beamerthemesplit}
\usepackage{graphicx}

\usepackage{tikz}


\mode<presentation>
{
  \usetheme{Warsaw}
  % or ...

  \setbeamercovered{transparent}
  % or whatever (possibly just delete it)
}


\title{IDP intellectual document processing}
\author{Max Talanov}
\date{\today}

\begin{document}

\frame{\titlepage}

\section[Outline]{}
\frame{\tableofcontents}

\section{Domain}
\frame
{
  \frametitle{Domain}
  80\% of information is unstructured. UnQueriable - this means that
    you have this information in your computer but you can not
    create any analysis based on it. This information lies as sheets
    of paper in documents of various formats.
}

\section{Solution}
\subsection{Technologies overview}
\frame{
  \frametitle{Technologies overview}
  \begin{itemize}
    \item Natural language processing
      \begin{description}
      \item [UIMA] IBM (Unstructured information management architecture)
      \item [GATE] University of Sheffield (General architecture for text
        engineering)
      \item [Minorthird] CMU (Toolkit for storing text, annotating text, and learning to extract entities and categorize text)
      \end{description}
    \item Machine learning
      \begin{description}
        \item [Weka] University of  Waikato (Weka is a collection of machine learning algorithms for solving real-world data mining problems.)
        \item [Rapid miner] rapid-i (Open-source data mining solution,
          built on base of Weka)
      \end{description}
  \end{itemize}
}

\subsection{Real life example}
\frame{
  Let's consider simple example of structuring the job offer text in
  XML format.

\emph{

Newsgroups: austin.jobs

Subject: COMPUTER TECHNITION NEEDED FOR RETAIL STORE 451-2489

Date: Thu, 28 Aug 1997 17:16:14 GMT

Organization: Jump Point Communications, Inc.

Message-ID: 3405a5f4.92436695@NEWS.JUMPNET.COM$>$

The computer and photo industries have merged into CompuImage
"where photos \& computers meet". This exciting concept holds
tremendous growth opportunitines. We are looking for a customer
friendly person expienced with the Internet, Networking, Windows 95,
Windows NT,  Foxpro, general troubleshooting and product titles. Call
Clifford @ 451-2489 or E-mail cliff@compu-image.com.

}

}

\subsection{Information structure}
\frame{

  We are going to get following structure of XML document:

  \begin{description}
  \item[root] for XML wellformness
    \begin{description}
    \item[header] The supplemental information section
      \begin{description}
      \item[newsgroup] austin.jobs
      \item[subject] COMPUTER TECHNITION NEEDED FOR RETAIL STORE 451-2489
      \item[post date] 28 Aug 1997
      \item[id] 3405a5f4.92436 ...
      \end{description}
    \item[body] Main part of an offer
    \end{description}
  \end{description}

}


\subsection{Information Extraction}
\frame{
  \frametitle{Information Extraction Description}

  We are going to use Information Extraction approach to
  structure the text document in XML. 

 There are tree parts of information extraction task: 
 \begin{itemize}
    \item Train (Create annotators)
    \item Test (Evaluate)
    \item Apply (Most important for us)
  \end{itemize}
  
  First of all we will train the annotators (programs that can mark
  proper parts of the text) on positive examples, then we are going
  to let them annotate plain text examples and put results in XML file.
  
%  \emph{ Start train_apply.en }
}
\section{Application}
\subsection{Application structure}
\frame {
  \frametitle{Application structure}
  \begin{center}
    \includegraphics[scale=0.45]{struct}
  \end{center}
}
\frame{
  \frametitle{Application structure description}
  \begin{description}
  \item[idp.Trainer] input is Annotated (XML) documents - training set
  \item[idp.Trainer] output is Annotators - serialized Java objects,
    that are learned to annotate
  \item[idp.Evaluator] input is Annotators, output is Precision,
    Recall, F-measure parameters
  \item[idp.Applier] input is Annotators and text to structure - test
    set
  \item[idp.Applier] output is structured documents
  \end{description}
}

\subsection{Result example}
\frame{
  $<$root$>$

  $<$\_predicted\_i\_header$>$

  Newsgroups:
  $<$\_predicted\_i\_newsgroup$>$austin.jobs$<$/\_predicted\_i\_newsgroup$>$

  Subject: $<$\_predicted\_i\_subject$>$COMPUTER TECHNITION NEEDED
  FOR RETAIL STORE 451-2489$<$/\_predicted\_i\_subject$>$

  Date: Thu, $<$\_predicted\_i\_post\_date$>$28 Aug
  1997$<$/\_predicted\_i\_post\_date$>$17:16:14 GMT

  Message-ID:
  $<$\_predicted\_i\_id$>$3405a5f4...$<$/\_predicted\_i\_id$>$

  $<$/\_predicted\_i\_header$>$

  $<$\_predicted\_i\_body$>$

  The computer and photo industries have merged into CompuImage
  "where photos \& computers meet". This exciting concept holds
  tremendous growth opportunitines. ...

  $<$/\_predicted\_i\_body$>$

  $<$/root$>$
  
}
\frame{
This looks much better, we can form analytical report about
number of specialist are required by company, skills that are required
average salary that is suggested during 1997 year. We can track
trends of average salary fluctuation for different skill set through
1997 year. All this information could be retrieved by XQuery for
example.
}
\section{Dictionary}
\frame{
  \begin{description}
  \item[Annotation] Process of marking up some parts of the text
    according to some rule(learned in case of Machine learning)
  \item [Annotator] Some program that makes annotation
  \end{description}
}

\end{document}
